\documentclass[a4paper]{article}
\usepackage{fontspec}
\usepackage[russian]{babel}
\usepackage{amsmath}
\usepackage{amssymb}

\usepackage[a4paper]{geometry}
\usepackage{indentfirst}
\usepackage{color}
\usepackage{graphicx}
\usepackage{caption}
\usepackage{subcaption}

\setmainfont{CMU Serif}
\setsansfont{CMU Sans Serif}

\usepackage[backend=biber]{biblatex}
\addbibresource{refs.bib}

\newcommand{\tbd}[1]{
    \textcolor{red}{#1}
}
\newcommand{\tworowcell}[2]{
    \begin{tabular}{@{}c@{}}#1 \\ #2\end{tabular}
}
\renewcommand{\setminus}{\mathbin{\backslash}}

\begin{document}

\title{Рассмотрение методов отбора признаков и эталонов на примере задачи распознавания телевизионной рекламы в видеопотоке}
\author{Антон Ларин\\ М0813-2 \and Сергей Ларин\\ М0816-2 \and Владислав Соврасов\\ М0813-2}
\date{\today}
\maketitle

\begin{abstract}
В данной работе производится обзор методов отбора признаков (feature selection) и отбора эталонов (instance selection), служащих для сокращения размера задачи машинного обучения. Приводятся описания методов PCA, отбора признаков на основе случайного леса и методов CNN, FCNN, CCIS отбора эталонов. Также обсуждаются результаты их применения к модельной задаче распознавания телевизионной рекламы в видеопотоке.
\end{abstract}

\section{Введение}
Задача распознавания рекламы в видео с развитием вычислительных средств становится всё более доступной. Она является актуальной как для компаний, занимающихся кабельным вещанием или его мониторингом, так и для простых пользователей, желающих записывать и хранить передачи без рекламы.
Одним из имеющихся на рынке средств, позволяющих решать задачу, является редактор Nero Vision, входящий в состав известного пакета Nero Suite. Неизвестно, какой алгоритм при этом используется, но для его работы достаточно обычного домашнего компьютера.
\par
С точки зрения машинного обучения эта задача является задачей бинарной классификации: необходимо пометить каждый кадр видеопотока как содержащий рекламу или нет. В данной работе будет рассмотрена модельная задача распознавания. На её примере планируется рассмотреть работу некоторых методов машинного обучения и техник сокращения объёма данных, требуемых для обучения.
\section{Описание задачи и методологии работы}
\subsection{Задача и набор данных}
Выбранная для тестирования методов задача заключается в определении, является ли элемент видеопотока телевизионной рекламой. Наша группа не занималась получением признакового описания из видеороликов, а лишь проводила эксперименты на готовом наборе данных, полученном исследователями в \cite{vyas}. Авторы указанной работы преобразовали 150 часов видео с 5 новостных каналов (BBC, CNN, CNN-IBN, NDTV, TIMES NOW) в признаковые описания коротких видеофрагментов, упорядоченных хронологически. Набор данных включает визуальные (длительность фрагмента, распределение разности кадров, распределение движения, распределение текста на экране и процент движущихся границ объектов на видео) и аудиопризнаки (спектральные характеристики аудиосигнала и bag of audio words). Каждому объекту в наборе данных присвоена метка класса: \(+1\) --- фрагмент является рекламой, \(-1\) --- не является. В таблице~\ref{table:class-distr} указаны соотношения классов в данных каждого из каналов:

\begin{table}
    \centering
    \begin{tabular}{|p{4cm}||c|c|c|c|c|}
    \hline
    Канал & BBC & CNN & CNN-IBN & TIMES NOW & NDTV \\ \hline
    Число объектов & 17720 & 22545 & 33117 & 39252 & 17051 \\ \hline
    \% объектов, соответствующих рекламе & 47.5 & 63.9 & 65.5 & 64 & 73.7\\
    \hline
    \end{tabular}
    \caption{Соотношение классов в наборе данных}
    \label{table:class-distr}
\end{table}

\subsection{Методология}
При рассмотрении методов отбора признаков и эталонов нас интересовало то, как их работа отражается на качестве классификации базовых методов машинного обучения и на времени, требующемся для их обучения. Исходя из этого, мы структурировали эксперименты следующим образом:
\begin{enumerate}
    \item Использование базовых методов машинного обучения\\
    Были проведены эксперименты с 5 базовыми методами (см.~\ref{sec:base-methods}) на данных с каждого видеоканала, с изменением одного параметра по некоторой сетке. Для каждого сочетания (метод, канал, значение параметра) была проведена 10-кратная кросс-валидация, в результате которой были получены среднее значение и стандартное отклонение для качества классификации и времени обучения. На основе результатов данной работы мы выбрали по одному значению параметров для каждого метода (из тех, которые имеют параметры) и использовали их в следующих частях работы.
    \item Использование методов отбора признаков\\\label{item:feature-selection}
    Были проведены эксперименты с некоторыми методами отбора признаков по схеме, аналогичной предыдущему пункту, с тем отличием, что в данном случае варьировался параметр метода отбора признаков.
    \item Использование методов отбора эталонов\\
    Аналогично п.~\ref{item:feature-selection}.
\end{enumerate}

\subsection{Используемые методы}\label{sec:base-methods}
Мы выбрали 5 методов машинного обучения в качестве базовых, в этом разделе приводятся их описания и результаты, показанные методами для разных параметров. Для проведения экспериментов мы использовали программные реализации методов из пакета scikit-learn \cite{sklearn}.

\subsubsection{\(k\) ближайших соседей}
\(k\)NN --- простой метрический метод, относящий \(\mathbf{x}\) в мажоритарный класс \(k\) ближайших (в нашем случае в евклидовой метрике) соседей из обучающей выборки. Он относится к категории методов так называемого \emph{ленивого обучения} --- фаза обучения у него состоит лишь в сохранении тестовой выборки полностью. При классификации он проходит по всем объектам обучающей выборки, вычисляя расстояния до классифицируемого объекта, что может выливаться в длительное время работы. Использованная нами реализация метода сохраняет обучающую выборку в KD-дерево, что несколько нивелирует данный негативный эффект. Известно, что \(k\)NN может показывать неудовлетворительные результаты в задачах с сильно многомерным пространством признаков, из-за того, что ближайшие к классифицируемому векторы все оказываются на примерно одинаковом расстоянии от него в евклидовой метрике \cite{beyer}. Также, ленивость \(k\)NN выливается в высокие требования к объёму памяти при работе с большими наборами данных. Эти особенности обуславливают наш интерес к методу \(k\)NN в рамках данной работы.

В тестовых запусках использовались значения параметра \(k\in\{1,3,5,\dotsc,101\}\). На рис.~\ref{fig:knn-base} представлены зависимости качества классификации и времени обучения от \(k\) для разных каналов.
\begin{figure}[h!]
    \centering
	\begin{subfigure}{0.45\textwidth}
		\includegraphics[width=\textwidth]{images/knn.png}
		\caption{Зависимости качества классификации от \(k\) для разных каналов.}\label{fig:knn-base-scores}
	\end{subfigure}
	\begin{subfigure}{0.45\textwidth}
		\includegraphics[width=\textwidth]{images/knnTime.png}
		\caption{Зависимости времён обучения от \(k\) для разных каналов.}
	\end{subfigure}
	\caption{Результаты для метода \(k\)NN}\label{fig:knn-base}
\end{figure}

За исключением аномального выброса на канале CNN-IBN метод \(k\)NN тратит примерно одно и то же время на обучение, пропорциональное объёму обучающей выборки. Это имеет место, потому что обучение \(k\)NN состоит в построении KD-дерева, а сложность этой операции зависит лишь от числа элементов в дереве, но не от \(k\). Таким образом, выбор \(k\) продиктован только качеством классификации. Исходя из рис.~\ref{fig:knn-base-scores} наилучшее значение \(k\) находится в промежутке \([13,21]\). Для дальнейших экспериментов было выбрано значение \(k=15\).

\subsubsection{Линейный дискриминантный анализ}
Линейный дискриминантный анализ в задаче бинарной классификации использует предположения о том, что объекты обоих классов распределены в пространстве признаков нормально, и что матрицы матрицы ковариации этих распределений совпадают. При данных предположениях уравнение разделяющей поверхности, получаемое из принципа максимума апостериорной вероятности, оказывается линейным. Таким образом, в задаче бинарной классификации LDA строит разделяющую гиперплоскость.

LDA в отличие от остальных рассматриваемых методов не имеет настраиваемых параметров. Результаты его работы представлены в сводной таблице~\ref{table:base-all}.

\subsubsection{Машина опорных векторов}
Данный метод является линейным классификатором, он строит гиперплоскость в пространстве признаков, имеющую максимальные зазоры до объектов каждого из классов. Пусть обучающая выборка имеет вид \(\left(X, Y\right)=\{\left(\mathbf{x}_i,y_i\right):i\in\{1,\dotsc,n\}\}\), классы линейно сепарабельны, и разделяющая гиперплоскость описывается \(\mathbf{w}^T\mathbf{x}-b=0\). Зазор ограничивается гиперплоскостями, параллельными искомой: \(\mathbf{w}^T\mathbf{x}-b=\pm1\). Максимизация зазора означает минимизацию \(\|\mathbf{w}\|\) при ограничениях \(y_i\left(\mathbf{w}^T\mathbf{x}_i-b\right)\geqslant1,\;i\in\{1,\dotsc,n\}\). Применением теоремы Куна -- Таккера данная задача сводится к задаче поиска седловой точки функции Лагранжа. При решении этой задачи получаются значения множителей Лагранжа \(\alpha_i, i\in\{1,\dotsc,n\}\), с помощью которых можно вычислить \(\mathbf{w}=\sum_{i=1}^n \alpha_i y_i\mathbf{x}_i\) и \(b=\mathbf{w}^T\mathbf{x}_j-y_j\), где \(j\) --- индекс одного из ненулевых множителей Лагранжа. Лишь несколько \(\alpha_i\) имеют положительное значение, и \(\mathbf{x}_i\), соответствующие таким \(\alpha_i\), называются \emph{опорными векторами}.

В общем случае гарантировать линейную сепарабельность объектов разных классов нельзя, и применяется алгоритм SVM с мягким зазором (soft-margin SVM) \cite{cortes-vapnik}. Отличие от случая линейной сепарабельности заключается во введении \(n\) дополнительных переменных \(\xi_i\), характеризующих ошибки классификации на соответствующих объектах. При этом ограничения принимают вид \(y_i\left(\mathbf{w}^T\mathbf{x}_i - b\right)\geqslant 1-\xi_i,\;i\in\{1,\dotsc,n\}\), а к целевой функции добавляется штраф \(C\sum_{i=1}^n\xi_i\), где \(C\) --- параметр мягкого зазора.

Зачастую параметр мягкого зазора \(C\) подбирается по логарифмической сетке, что мы и проделали. На рис.~\ref{fig:svm-base} представлены результаты данного эксперимента. Из них видно, что среди опробованных значений \(C\) на 4 каналах из 5 наилучший результат SVM показал при \(C=4\), именно это значение было выбрано для дальнейших экспериментов с отбором признаков и эталонов. Данный результат был получен при использовании RBF-ядра, которое является ядром по умолчанию в реализации SVM в scikit-learn. В ходе неструктурированных экспериментов было обнаружено, что линейное ядро требует на порядок больше времени в нашей задаче и показывает худший результат на \(1-2\%\), а также, забегая вперёд, можно сказать, что линейное ядро показало ухудшение результатов при снижении размерности данных, в то время как результаты ядра RBF улучшились.

\begin{figure}[h!]
    \centering
	\begin{subfigure}{0.45\textwidth}
		\includegraphics[width=\textwidth]{images/svm.png}
		\caption{Зависимости качества классификации от \(C\) для разных каналов.}
	\end{subfigure}
	\begin{subfigure}{0.45\textwidth}
		\includegraphics[width=\textwidth]{images/svmTime.png}
		\caption{Зависимости времён обучения от \(C\) для разных каналов.}
	\end{subfigure}
	\caption{Результаты для метода SVM}\label{fig:svm-base}
\end{figure}

\subsubsection{Случайный лес}
Данный алгоритм является расширением бэггинга для деревьев решений. При использовании бэггинга \(B\) деревьев решений тренируются каждое на своём подмножестве обучающей выборки \(\left(X_b, Y_b\right),\;b=1,\dotsc,B\), полученном из исходной обучающей выборки случайным выбором с повторениями. Ответом классификатора будет класс, за который проголосовало больше всего деревьев. Такое построение классификатора позволяет получить некоррелированные деревья, благодаря чему голосование решает проблемы одиночных деревьев решений с переобучением, вызванные их высокой чувствительностью к выбросам в обучающей выборке. Случайный лес развивает идею бэггинга на шаг дальше. В этом методе модифицируется алгоритм построения дерева: при поиске правила для каждой новой вершины используется случайное подмножество признаков. Если существует сильная зависимость между некоторым признаком и классом объекта, то достаточно большое количество деревьев могут выбрать этот признак при присвоении решающего правила вершине дерева. Таким образом, случайный выбор подмножества признаков позволяет ещё сильнее сократить чувствительность деревьев к шумам.

Были проведены тестовые запуски данного метода для значений \(B\in\{5,10,\dotsc,155\}\). На рис.~\ref{fig:randfor-base} изображены зависимости качества классификации и времени обучения от числа деревьев. Время, требуемое для обучения случайного леса, пропорционально числу деревьев, что мотивирует выбор как можно меньшего \(B\), обеспечивающего приемлемое качество классификации. В данном случае нами было выбрано \(B=60\).

\begin{figure}[h!]
    \centering
	\begin{subfigure}{0.45\textwidth}
		\includegraphics[width=\textwidth]{images/randfor.png}
		\caption{Зависимости качества классификации от числа деревьев для разных каналов.}
	\end{subfigure}
	\begin{subfigure}{0.45\textwidth}
		\includegraphics[width=\textwidth]{images/randforTime.png}
		\caption{Зависимости времён обучения для разных каналов.}
	\end{subfigure}
	\caption{Результаты для случайного леса}\label{fig:randfor-base}
\end{figure}

\subsubsection{Градиентный бустинг деревьев решений}
Бустинг --- мета-алгоритм, который итеративно добавляет в модель <<слабые>> подмодели (модель, предсказание которой слабо коррелирует с истинной классификацией, но показывает лучший результат, чем случайный выбор класса). Градиентный бустинг добавляет подмодели таким образом, чтобы минимизировать некоторую дифференцируемую функцию потерь. Пусть к началу \(m\)-ой итерации ГБ построил модель, обладающую решающим правилом \(F_{m-1}(\mathbf{x})\). Если он выберет \(m\)-ую модель идеально, то \(F_{m}(\mathbf{x})=F_{m-1}(\mathbf{x}) + h_m(\mathbf{x})=y\), или иначе \(h_m(\mathbf{x})=y-F_{m-1}(\mathbf{x})\). Таким образом, ГБ на каждой итерации обучает новую подмодель на выборке \(\left(X, \{y_i-F_{m-1}(\mathbf{x}_i):y_i\in Y\}\right)\). Для квадратичной функции потерь \(L\left(y, F\right)=\frac12 \left(y-F(\mathbf{x})\right)^2\) остатки \(y-F_{m-1}(\mathbf{x})\) есть антиградиенты функции потерь, и метод, по сути, является градиентным спуском. Аналогично, для произвольной дифференцируемой функции потерь обучение \(m\)-ой подмодели на выборке с остатками вместо меток классов является реализацией градиентного спуска относительно функции потерь. Кроме этого, ГБ вычисляет коэффициент для решающего правила \(h_m(\mathbf{x})\), минимизируя функцию потерь вдоль направления антиградиента, и таким образом реализует наискорейший градиентный спуск.

Градиентый бустинг деревьев решений является реализацией ГБ для подмоделей в виде деревьев решений. Возможна модификация метода, которая позволяет улучшить качество классификации отдельных деревьев: вместо одного коэффициента для каждого решающего правила \(h_m(\mathbf{x})\) подбирать коэффициенты для каждого листа дерева. Такой алгоритм называется TreeBoost.

Для предотвращения переобучения подмоделей в градиентном бустинге могут применяться техники регуляризации: подбор числа итераций, модификация обновления решающего правила модели с помощью введения коэффициента \(\nu\) перед \(h_m(\mathbf{x})\), меньшего 1, в некотором смысле замедляющего построение модели (shrinkage).

Результаты запусков с различным числом деревьев представлены на рис.~\ref{fig:gtb-base}. Число деревьев \(M\) выбиралось из множества \(\{50,60,\dotsc,150\}\).
\begin{figure}[h!]
    \centering
	\begin{subfigure}{0.45\textwidth}
		\includegraphics[width=\textwidth]{images/gtb.png}
		\caption{Зависимости качества классификации от числа деревьев для разных каналов.}
	\end{subfigure}
	\begin{subfigure}{0.45\textwidth}
		\includegraphics[width=\textwidth]{images/gtbTime.png}
		\caption{Зависимости времён обучения для разных каналов.}
	\end{subfigure}
	\caption{Результаты для градиентного бустинга деревьев решений}\label{fig:gtb-base}
\end{figure}

Бустинг --- врождённо последовательный алгоритм, поэтому время обучения у него пропорционально числу поддеревьев. В данном случае выбранная нами сетка параметра не позволила найти экстремум качества классификации. Используемое в дальнейшем число деревьев \(M=100\) было выбрано в качестве компромисса между качеством классификации и затратами времени на обучение модели в ситуации ограниченности вычислительных ресурсов.

\subsubsection{Результаты для выбранных параметров}
В результате использования методов, описанных в данном разделе, были выбраны параметры, с которыми они будут запускаться после отбора признаков или эталонов. А именно, далее будут использоваться метод 15 ближайших соседей, случайный лес с 60 деревьями, SVM с \(C=4\) и градиентный бустинг со 100 деревьями. В таблице~\ref{table:base-all} сведена вся информация о качестве классификации и производительности методов для выбранных значений параметров. \(Q\) обозначает среднее качество классификации в результате кросс-валидации, \(T_{tr}\) --- среднее время обучения.

\begin{table}[h!]
    \centering
    \begin{tabular}{|c||c|c|c|}
		\cline{2-4}
		\multicolumn{1}{c||}{} & \(k\)NN & LDA & SVM \\
		\hline \hline
		\input{base-table1.txt}
    \end{tabular}
\newline \vspace*{0.5cm} \newline
	\begin{tabular}{|c||c|c|}
		\cline{2-3}
		\multicolumn{1}{c||}{} & Random forest & GTB \\
		\hline \hline
		\input{base-table2.txt}
	\end{tabular}
    \caption{Сводная таблица результатов базовых методов}
    \label{table:base-all}
\end{table}


\section{Отбор признаков}
Основные цели отбора признаков --- уменьшение размерности задачи, а также выявление и отбрасывание признаков, не несущих полезной для решения задачи информации. Снижение размерности также ведёт к уменьшению времени обучения алгоритмов и уменьшению объёма памяти, занимаемого тренировочной выборкой.
 \par
Для рассматриваемого набора данных большая часть переменных не меняет значений на всех объектах выборки. Такая ситуация произошла из-за использования схемы Bag of Words при создании датасета. Поэтому в качестве предобработки данных все признаки с нулевой дисперсией были отброшены. В результате размерность задачи была сокращена с 4125 до около 230 переменных (227--229 в зависимости от канала). 
\subsection{PCA}
Для дальнейшего уменьшения количества признаков использовался метод главных компонент (principal component analysis или PCA) \cite{pearson}. Этот метод позволяет снизить размерность оптимальным с точки зрения некоторого критерия образом. Рассмотрим его подробнее.
 \par
 Постановка задачи, решаемой в методе PCA звучит следующим образом: найти линейное многообразие заданной  размерности, сумма квадратов расстояний до которого от данной системы точек минимальна. Стоит обратить внимание на то, что информация о принадлежности точек к какому-либо классу здесь не учитывается. Пусть \( X_i \subseteq \mathbb{R}^n, i=\overline{1,m} \) --- набор произвольных векторов. Требуется найти векторы \(v_0\ldots v_d\subseteq \mathbb{R}^d, \Vert v_j \Vert = 1, v_i \bot v_j, j \neq  i, d \leq n\), задающие линейное многообразие \(L\) так, чтобы \( \sum_{i=1}^{m} dist^2(X_i, L) = \sum_{i=1}^{m}\Vert X_i - v_0 - \sum_{j=1}^k\langle X_i - v_0; v_j\rangle\ v_j \Vert\rightarrow \mathrm{min}\). В качестве редуцированных данных выступают проекции векторов \( X_i\) на многообразие \(L\). 
  
 \subsubsection*{Результаты применения PCA}
 Эксперименты по подбору оптимальной размерности проводились следующим образом: перебирались все значения 
 размерности редуцированных данных начиная от 10 с шагом в 10 и проводилась кросс-валидация с параметром разбиения 10. Эта процедура была проведена для каждого канала и для каждого метода. Параметры методов 
 взяты из секции~\ref{sec:base-methods}.
\par
Метод \(k\) ближайших соседей оказался нечувствителен к снижению размерности данных. Качество классификации не меняется, возможно, из-за того, что PCA в условиях данной задачи достаточно хорошо сохраняет локальную структуру пространства признаков. Время обучения метода \(k\)NN мало, поэтому PCA
имеет смысл применять для уменьшения объёма данных, которые хранит метод для своей работы.

\begin{figure}[h!]
    \centering
	\begin{subfigure}{0.45\textwidth}
		\includegraphics[width=\textwidth]{images/PCA-kNN.png}
		\caption{Зависимости качества классификации от размерности PCA}
	\end{subfigure}
	\begin{subfigure}{0.45\textwidth}
		\includegraphics[width=\textwidth]{images/PCA-kNNTime.png}
		\caption{Зависимости времён обучени от размерности PCA}
	\end{subfigure}
	\caption{Результаты для метода \(k\)NN}\label{fig:knn_pca}
\end{figure}

\par
Линейный дискриминантный анализ в сочетании с PCA показал ухудшение результатов. Время обучения этого метода мало, так что редукция размерности в данном случае оказалась бесполезной.

\begin{figure}[h!]
    \centering
	\begin{subfigure}{0.45\textwidth}
		\includegraphics[width=\textwidth]{images/PCA-LDA.png}
		\caption{Зависимости качества классификации от размерности PCA}
	\end{subfigure}
	\begin{subfigure}{0.45\textwidth}
		\includegraphics[width=\textwidth]{images/PCA-LDATime.png}
		\caption{Зависимости времён обучения от размерности PCA}
	\end{subfigure}
	\caption{Результаты для метода LDA}\label{fig:lda_pca}
\end{figure}

\par
Наилучшего результата с точки зрения качества классификации удалось достичь с помощью машины опорных векторов и снижения размерности до 50 или 60, в зависимости от канала. Кроме того, при меньшей размерности значительно падает время работы SVM.

\begin{figure}[h!]
    \centering
	\begin{subfigure}{0.45\textwidth}
		\includegraphics[width=\textwidth]{images/PCA-SVM.png}
		\caption{Зависимости качества классификации от размерности PCA}
	\end{subfigure}
	\begin{subfigure}{0.45\textwidth}
		\includegraphics[width=\textwidth]{images/PCA-SVMTime.png}
		\caption{Зависимости времён обучения от размерности PCA}
	\end{subfigure}
	\caption{Результаты для метода SVM}\label{fig:svm_pca}
\end{figure}

\par
Хорошие результаты при уменьшении размерности с помощью PCA показал алгоритм random forest. С уменьшением размерности задачи растёт доля верных предсказаний (на рис.~\ref{fig:randfor_pca} показаны графики зависимости качества обучения алгоритма на разных каналах). При одновременном улучшении результата кросс-валидации практически линейно уменьшается время обучения алгоритма. Оптимальное число признаков для алгоритма random forest с точки зрения сочетания качества обучения и затраченного времени --- 50 или 60 в зависимости от канала.
 
\begin{figure}[h!]
    \centering
	\begin{subfigure}{0.45\textwidth}
		\includegraphics[width=\textwidth]{images/PCA-randfor.png}
		\caption{Зависимости качества классификации от размерности PCA}
	\end{subfigure}
	\begin{subfigure}{0.45\textwidth}
		\includegraphics[width=\textwidth]{images/PCA-randforTime.png}
		\caption{Зависимости времён обучения от размерности PCA}
	\end{subfigure}
	\caption{Результаты для метода random forest}\label{fig:randfor_pca}
\end{figure} 
 
\par
Ещё одним методом, слабо чувствительным к применению PCA стал градиентый бустинг деревьев решений. PCA в таком случае можно использовать чтобы снизить время обучения модели. Из рис.~\ref{fig:gtb_pca} можно увидеть, что при снижении размерности данных в 2 раз время обучения падает тоже примерно в 2 раза.

\begin{figure}[h!]
    \centering
	\begin{subfigure}{0.45\textwidth}
		\includegraphics[width=\textwidth]{images/PCA-GTB.png}
		\caption{Зависимости качества классификации от размерности PCA}
	\end{subfigure}
	\begin{subfigure}{0.45\textwidth}
		\includegraphics[width=\textwidth]{images/PCA-GTBTime.png}
		\caption{Зависимости времён обучения от размерности PCA}
	\end{subfigure}
	\caption{Результаты для метода Gradient tree boosting}\label{fig:gtb_pca}
\end{figure}

\par
\subsection{Отбор признаков с помощью ансамблей деревьев решений}
Идея о том, что с помощью случайного леса можно для каждого признака оценить его важность была высказана в статье Бреймана \cite{breiman}.
\par
Первые шаг в оценке важности переменной в тренировочной выборке --- обучение случайного леса на этих данных. Во время процесса построения модели для каждого элемента тренировочного набора вычисляется out-of-bag-ошибка.
Out-of-bag-ошибка --- это доля примеров обучающей выборки, неправильно классифицируемых лесом, если не учитывать голоса деревьев на примерах, входящих в их собственную бутстрап подвыборку.
Для того, чтобы оценить важность \(j\)-го признака после тренировки, значения \(j\)-го параметра перемешиваются для всех записей тренировочного набора и out-of-bag-ошибка считается снова. Важность признака оценивается путем усреднения по всем деревьям разности показателей out-of-bag-ошибок до и после перемешивания значений. При этом значения таких ошибок нормализуются на среднеквадратическое отклонение разностей.
\par
После этого осуществляется отсечение признаков с низким весом по заданному порогу. Можно также выбирать \(k\) самых важных признаков, что и будет делаться далее.

\subsubsection*{Результаты отбора признаков с помощью случайного леса}
В качестве классификатора для оценки важности признаков выступает случайный лес из 60-ти деревьев. Схема экспериментов такая же, как с PCA. Из-за недостатка времени эксперименты не были проведены полностью (результаты методов SVM и GTB получены для двух каналов), однако их достаточно, чтобы сказать, что отбор по важности проявил себя не хуже, чем PCA.
\par
На некоторых каналах метода \(k\) ближайших соседей показывает нехарактерно высокое для него качество классификации при маленьком количестве признаков (рис.~\ref{fig:knn_rfs}). Скорее всего, это случайность, так как при дальнейшем увеличении количества признаков качество классификации резко падает и стабилизируется на значении, которое было до редукции размерности.

\begin{figure}[h!]
    \centering
	\begin{subfigure}{0.45\textwidth}
		\includegraphics[width=\textwidth]{images/RFS-kNN.png}
		\caption{Зависимости качества классификации от количества признаков}
	\end{subfigure}
	\begin{subfigure}{0.45\textwidth}
		\includegraphics[width=\textwidth]{images/RFS-kNNTime.png}
		\caption{Зависимости времён обучения от количества признаков}
	\end{subfigure}
	\caption{Результаты для метода \(k\)NN}\label{fig:knn_rfs}
\end{figure}

\par
Как и в случае с PCA, качество классификации линейного дискриминантного анализа монотонно ухудшается при снижении размерности данных (рис.~\ref{fig:lda_rfs}). Применение отбора в этом случае бесперспективно, так как время обучения метода мало даже для нередуцированных данных.

\begin{figure}[h!]
    \centering
	\begin{subfigure}{0.45\textwidth}
		\includegraphics[width=\textwidth]{images/RFS-LDA.png}
		\caption{Зависимости качества классификации от количества признаков}
	\end{subfigure}
	\begin{subfigure}{0.45\textwidth}
		\includegraphics[width=\textwidth]{images/RFS-LDATime.png}
		\caption{Зависимости времён обучения от количества признаков}
	\end{subfigure}
	\caption{Результаты для метода LDA}\label{fig:lda_rfs}
\end{figure}

\par
Результаты, полученные на части данных, показывают, что качество классификации SVM можно повысить путём отбора признаков с помощью случайного леса. Оптимальное значение количества признаков --- 100 или 110 в зависимости от канала. Время обучения при такой редукции падает примерно в два раза.

\begin{figure}[h!]
    \centering
	\begin{subfigure}{0.45\textwidth}
		\includegraphics[width=\textwidth]{images/RFS-SVM.png}
		\caption{Зависимости качества классификации от количества признаков}
	\end{subfigure}
	\begin{subfigure}{0.45\textwidth}
		\includegraphics[width=\textwidth]{images/RFS-SVMTime.png}
		\caption{Зависимости времён обучения от количества признаков}
	\end{subfigure}
	\caption{Результаты для метода SVM}\label{fig:svm_rfs}
\end{figure}

\par
Отбор признаков практически не влияет на качество классификации случайного леса (рис.~\ref{fig:randfor_rfs}), поэтому делать это имеет  смысл для уменьшения времени работы метода или объема информации, занимаемого тренировочной выборкой.

\begin{figure}[h!]
    \centering
	\begin{subfigure}{0.45\textwidth}
		\includegraphics[width=\textwidth]{images/RFS-randforest.png}
		\caption{Зависимости качества классификации от количества признаков}
	\end{subfigure}
	\begin{subfigure}{0.45\textwidth}
		\includegraphics[width=\textwidth]{images/RFS-randforestTime.png}
		\caption{Зависимости времён обучения от количества признаков}
	\end{subfigure}
	\caption{Результаты для метода random forest}\label{fig:randfor_rfs}
\end{figure} 

\par
Ситуация с градиентным бустингом деревьев решений очень похожа на ситуацию со случайным лесом: метод малочувствителен к редукции размерности в смысле качества классификации (рис.~\ref{fig:gtb_rfs}).

\begin{figure}[h!]
    \centering
	\begin{subfigure}{0.45\textwidth}
		\includegraphics[width=\textwidth]{images/RFS-GTB.png}
		\caption{Зависимости качества классификации от количества признаков}
	\end{subfigure}
	\begin{subfigure}{0.45\textwidth}
		\includegraphics[width=\textwidth]{images/RFS-GTBTime.png}
		\caption{Зависимости времён обучения от количества признаков}
	\end{subfigure}
	\caption{Результаты для метода Gradient tree boosting}\label{fig:gtb_rfs}
\end{figure}
\par

\section{Отбор эталонов}
Как правило, работа с выборками большого объёма сопряжена с большими затратами времени на обучение модели. Кроме этого, в достаточно популярном методе \(k\) ближайших соседей обучающая выборка хранится полностью, что в случае выборки большого объёма является ограничивающим для него фактором. Таким образом, привлекательными кажутся техники, которые позволяют уменьшить объём обучающей выборки, почти не теряя обобщающей способности. Данная тематика в литературе на английском языке называется instance selection (prototype selection).

В \cite{ps-taxonomy} было проведено крупномасштабное исследование, а также классификация методов отбора объектов обучающей выборки. Авторы разбили их на следующие группы по механизму работы:
\begin{itemize}
    \item методы сгущения (condensation) --- стремятся сократить число точек, далёких от границ классов, в предположении, что они слабо влияют на геометрию границы. Они стремятся сохранить качество классификации на обучающей выборке, при этом качество классификации на тестовой выборке может пострадать. Тем не менее, они, как правило, достигают высокой степени сокращения объёма обучающей выборки;
    \item методы редактирования (edition) --- стремятся сократить число точек, близких к границам классов, несогласованных с соседними, шумовых точек. Данные методы создают более гладкие границы между классами, и приводят к повышению обобщающей способности классификатора, но они в меньшей степени сокращают объём выборки, чем методы предыдущей группы;
    \item гибридные методы (hybrid) --- стремятся найти подмножество выборки как можно меньшей мощности, которое улучшает обобщающую способность на тестовых данных.
\end{itemize}

Столкнувшись со значительными затратами вычислительных ресурсов при тестировании различных методов машинного обучения в нашей задаче, мы решили опробовать несколько методов отбора эталонов и оценить, насколько большую пользу они могут принести в данной задаче. В пакете scikit-learn данный класс методов не представлен, а существующие реализации на Python \cite{scikit-protopy} показали себя неудовлетворительно с точки зрения производительности. Поэтому для проведения данной работы методы condensed nearest neighbor (CNN), fast condensed nearest neigbor (FCNN) и class conditional instance selection (CCIS), реализованные авторами \cite{ps-taxonomy}, были портированы на язык C. В следующих подразделах выбранные методы отбора эталонов описаны и приведены результаты, показанные ими на нашей задаче.

Все представленные в данном разделе методы используют \(k\)NN для внутренних нужд: оценки подмножеств обучающей выборки, нахождения новых элементов для добавления в неё или удаления из неё. Значительное количество методов, исследованных в \cite{ps-taxonomy} опираются на \(k\)NN (в частности, из-за того, что они чаще всего применяются в связке с \(k\)NN классификатором). Тем не менее, спектр методов отбора эталонов ими не ограничивается, часть методов в \cite{ps-taxonomy} относится к классу эволюционных. Некоторые из них могут агрессивно редуцировать обучающую выборку и обеспечивать хорошое качество классификации. К сожалению, данные методы требуют значительно больших затрат времени, чем рассматриваемые далее, что не позволило нам оценить их в рамках данной работы.

\subsection{Condensed nearest neighbor}
Данный метод хронологически является одним из первых методов отбора эталонов. Он был описан в 1966 году в \cite{hart} вместе с понятием \emph{согласованного подмножества} (consistent subset) обучающей выборки \(T\) --- такого подмножества \(S\subseteq T\), что метод ближайшего соседа, обученный на \(S\), правильно классифицирует \(T\setminus S\). Он представляет из себя простейший метод сгущения и оперирует с двумя множествами: \(S\) и \(R\). Изначально \(S\) содержит лишь первый элемент обучающей выборки, а \(R=\varnothing\). Второй элемент классифицируется методом ближайшего соседа на основе множества \(S\), и, если ему присвоен верный класс, то он добавляется в \(R\), иначе он добавляется в \(S\). Каждый следующий элемент \(T\) обрабатывается аналогичным образом. После первого прохода через \(T\) процедура начинает совершать проходы через множество \(R\), до тех пор, пока за весь проход через \(R\) ни один элемент не будет перенесён в \(S\), после чего в \(S\) находится согласованное подмножество \(T\).

Метод CNN был сформулирован для метода одного ближайшего соседа, однако расширение на \(k\) ближайших соседей производится очевидным образом: при классификации очередного элемента \(T\) (а впоследствии \(R\)) используется метод \(k\)NN с обучающей выборкой \(S\). Стоит отметить, что данный метод строит согласованное подмножество только по отношению к методу \(k\) ближайших соседей (для конкретного \(k\)), результирующее подмножество \(S\) не обязано быть согласованным для произвольного алгоритма классификации.

\subsubsection*{Результаты применения CNN}
Были проведены эксперименты с методом CNN для \(k\in\{1, 5, 10\}\). На рис.~\ref{fig:cnn-stats} указаны результаты, показанные непосредственно CNN: время работы и доля отобранных объектов. Оказалось, что в нашей задаче доля отобранных объектов не имеет ярко выраженной зависимости от \(k\). Время работы ожидаемо зависит от \(k\) линейно (в реализации был использован наивный алгоритм \(k\)NN с перебором всех соседей).

\begin{figure}[h!]
    \centering
	\begin{subfigure}{0.45\textwidth}
		\includegraphics[width=\textwidth]{images/cnn-stats.png}
		\caption{Доля отобранных объектов.}
	\end{subfigure}
	\begin{subfigure}{0.45\textwidth}
		\includegraphics[width=\textwidth]{images/cnn-TimeStats.png}
		\caption{Время, потраченное на отбор эталонов.}
	\end{subfigure}
	\caption{Результаты работы CNN.}\label{fig:cnn-stats}
\end{figure}

На рис.~\ref{fig:cnn-knn-results}--\ref{fig:cnn-gtb-results} приведены результаты работы базовых методов на обучающих выборках, редуцированных с помощью CNN. При использовании \(k\), отличных от 1, CNN выбирает примерно такое же количество эталонов, как и при \(k=1\), но сами эталоны оказываются хуже по качеству. Этот результат не выглядит неожиданным, если учесть, что CNN, будучи методом сгущения, стремится оставить объекты, близкие к границе классов --- объекты, у которых много соседей из другого класса. За исключением каналов TIMES NOW и CNN-IBN для метода SVM использование \(k>1\) также не приносит ускорения базового метода классификации, в основном, потому что \(\left|S\right|\) получается примерно такой же. Иными словами, в данной задаче нет смысла использовать CNN с \(k>1\).

Тот факт, что CNN строит согласованное подмножество для \(k\)NN, наводит на вопрос: насколько хороший результат покажет 15-NN на выходе CNN с \(k=15\)? Такой эксперимент был проведён с одним каналом NDTV и он привёл к неудовлетворительным результатам: качество классификации оказалось близким к случаю \(k=10\) (около 25\%).

\begin{figure}[h!]
    \centering
	\begin{subfigure}{0.45\textwidth}
		\includegraphics[width=\textwidth]{images/cnn-KNN.png}
		\caption{Качество классификации.}
	\end{subfigure}
	\begin{subfigure}{0.45\textwidth}
		\includegraphics[width=\textwidth]{images/cnn-KNNTime.png}
		\caption{Время обучения.}
	\end{subfigure}
	\caption{Результаты применения CNN для 15-NN.}\label{fig:cnn-knn-results}
\end{figure}

\begin{figure}[h!]
	\centering
	\begin{subfigure}{0.45\textwidth}
		\includegraphics[width=\textwidth]{images/cnn-LDA.png}
		\caption{Качество классификации.}
	\end{subfigure}
	\begin{subfigure}{0.45\textwidth}
		\includegraphics[width=\textwidth]{images/cnn-LDATime.png}
		\caption{Время обучения.}
	\end{subfigure}
	\caption{Результаты применения CNN для LDA.}\label{fig:cnn-lda-results}
\end{figure}

\begin{figure}[h!]
	\centering
	\begin{subfigure}{0.45\textwidth}
		\includegraphics[width=\textwidth]{images/cnn-SVM.png}
		\caption{Качество классификации.}
	\end{subfigure}
	\begin{subfigure}{0.45\textwidth}
		\includegraphics[width=\textwidth]{images/cnn-SVMTime.png}
		\caption{Время обучения.}
	\end{subfigure}
	\caption{Результаты применения CNN для SVM.}\label{fig:cnn-svm-results}
\end{figure}

\begin{figure}[h!]
	\centering
	\begin{subfigure}{0.45\textwidth}
		\includegraphics[width=\textwidth]{images/cnn-randforest.png}
		\caption{Качество классификации.}
	\end{subfigure}
	\begin{subfigure}{0.45\textwidth}
		\includegraphics[width=\textwidth]{images/cnn-randforestTime.png}
		\caption{Время обучения.}
	\end{subfigure}
	\caption{Результаты применения CNN для Random forest.}\label{fig:cnn-rf-results}
\end{figure}

\begin{figure}
	\centering
	\begin{subfigure}{0.45\textwidth}
		\includegraphics[width=\textwidth]{images/cnn-gradboosting.png}
		\caption{Качество классификации.}
	\end{subfigure}
	\begin{subfigure}{0.45\textwidth}
		\includegraphics[width=\textwidth]{images/cnn-gradboostingTime.png}
		\caption{Время обучения.}
	\end{subfigure}
	\caption{Результаты применения CNN для GTB}\label{fig:cnn-gtb-results}
\end{figure}

В таблице~\ref{table:cnn-results} приведены результаты применения метода CNN при \(k=1\), как наиболее эффективного варианта. \(R\) и \(T_{PS}\) обозначены процент объектов выборки, оставленный методом CNN, и время работы CNN, соответственно. Как и в таблице~\ref{table:base-all}, \(Q\) означает качество классификации (в скобках указано абсолютное значение, на которое \(Q\) изменилось по сравнению с соответствующим значением в таблице~\ref{table:base-all}), и \(T_{tr}\) означает время обучения (в скобках указано во сколько раз время обучения после редукции больше времени обучения на всей выборке).

Из таблицы~\ref{table:cnn-results} видно, что несмотря на то, что \(k=1\) --- наилучший параметр из опробованных, метод CNN сильно портит качество классификации базовых методов. Объём выборки сокращается в 2--3 раза, и это приводит к ощутимому ускорению <<тяжёлых>> базовых методов --- SVM (ускорение от 4 до 10 раз) и градиентного бустинга (ускорение примерно в 3 раза). На некоторых каналах потери качества классификации не выглядят катастрофическими (связки случайный лес+CNN-IBN, случайный лес+NDTV, градиентный бустинг+NDTV), а вместе с полученными в этих случаях ускорениями, можно считать, что данные пары получили пользу от отбора эталонов. Самым неудобным для CNN оказался канал BBC --- от него осталась почти половина объектов и качество классификации базовыми методами на нём пострадало больше всего.

\begin{table}[h!]
    \centering
    \begin{tabular}{|c||c||c|c|}
		\cline{2-4}
		\multicolumn{1}{c||}{} & Сжатие \(T\) & \(k\)NN & LDA \\
		\hline \hline
		\input{cnn-table1.txt}
	\end{tabular}
	\newline \vspace*{0.5cm} \newline
	\centering
	\begin{tabular}{|c||c|c|c|}
		\cline{2-4}
		\multicolumn{1}{c||}{} & SVM & Random forest & GTB \\
		\hline \hline
		\input{cnn-table2.txt}
    \end{tabular}
    \caption{Сводная таблица результатов метода CNN и базовых методов после применения CNN}
    \label{table:cnn-results}
\end{table}

\subsection{Fast condensed nearest neighbor}
Метод FCNN \cite{angiulli} был предложен с целью исправить некоторые недостатки CNN и других методов, основанных на \(k\)NN: зависимость результирующего подмножества \(S\) от порядка элементов в обучающей выборке и низкая производительность и масштабируемость. Он, как и CNN, является методом сгущения.

FCNN работает следующим образом. Изначально \(S\) содержит точки, ближайшие к барицентрам классов (барицентр класса --- \(\mathbf{x}_C=|C|^{-1}\sum_{c\in C}\mathbf{x}_c\), где \(C\) --- множество индексов объектов, принадлежащих одному классу). Затем, множество \(T\setminus S\) разбивается на \(|S|\) непересекающихся классов \(Vor(p, S, T)\) (так называемые Voronoi cells), в каждом из которых находятся объекты, для которых ближайшим соседом является один и тот же объект \(p\in S\). Затем все объекты из \(T\setminus S\) классифицируются методом ближайшего соседа с обучающей выборкой \(S\), и в каждом \(Vor(p, S, T)\) выбираются неверно классифицированные объекты, которые составляют множества \(Voren(p, S, T)\) (Voronoi enemies). После этого для каждого такого \(p\in S\), что \(Voren(p, S, T)\neq\varnothing\) выбирается ближайший к \(p\) объект из \(Voren(p, S, T)\) и добавляется в \(S\). На каждой следующей итерации процедура повторяется для нового \(S\). Алгоритм останавливается, когда на последней итерации в \(S\) не было добавлено ни одного элемента.

FCNN может быть расширен на число ближайших соседей, отличных от 1, аналогично CNN: при классификации объектов в \(T\setminus S\) нужно использовать метод \(k\)NN. В \cite{angiulli} приведено доказательство того, что он строит согласованное подмножество \(S\), одинаковое для любого упорядочивания обучающей выборки.

\subsubsection*{Результаты применения FCNN}
Аналогично CNN, с FCNN были проведены эксперименты для \(k\in\{1,5,10\}\). Доли отобранных объектов и времена работы для каждого из каналов приведены на рис.~\ref{fig:fcnn-stats}. В данном случае очевидно полное преимущество \(k=1\) над другими значениями параметра в смысле результатов FCNN.
\begin{figure}[h!]
    \centering
	\begin{subfigure}{0.45\textwidth}
		\includegraphics[width=\textwidth]{images/fcnn-stats.png}
		\caption{Доля отобранных объектов.}
	\end{subfigure}
	\begin{subfigure}{0.45\textwidth}
		\includegraphics[width=\textwidth]{images/fcnn-TimeStats.png}
		\caption{Время, потраченное на отбор эталонов.}
	\end{subfigure}
	\caption{Результаты работы FCNN.}\label{fig:fcnn-stats}
\end{figure}

На рис.~\ref{fig:fcnn-knn-results}--\ref{fig:fcnn-gtb-results} приведены результаты работы базовых методах на обучающих выборках, редуцированных с помощью FCNN. С этим методом ситуация получается противоположная CNN --- меньший \(k\) обеспечивает большую редукцию, но при этом базовые методы после редукции дают похожие качества классификации для всех \(k\). Одно из объяснений этому --- за счёт более сложной эвристики FCNN отбирает более качественные эталоны, что нивелирует разницу для базовых методов. С другой стороны, сама процедура FCNN оказывается более подвержена ошибкам классификации на границе классов для неединичных \(k\), поэтому она выполняет больше итераций.

\begin{figure}[h!]
    \centering
	\begin{subfigure}{0.45\textwidth}
		\includegraphics[width=\textwidth]{images/fcnn-KNN.png}
		\caption{Качество классификации.}
	\end{subfigure}
	\begin{subfigure}{0.45\textwidth}
		\includegraphics[width=\textwidth]{images/fcnn-KNNTime.png}
		\caption{Время обучения.}
	\end{subfigure}
	\caption{Результаты применения FCNN для 15-NN.}\label{fig:fcnn-knn-results}
\end{figure}

\begin{figure}[h!]
	\centering
	\begin{subfigure}{0.45\textwidth}
		\includegraphics[width=\textwidth]{images/fcnn-LDA.png}
		\caption{Качество классификации.}
	\end{subfigure}
	\begin{subfigure}{0.45\textwidth}
		\includegraphics[width=\textwidth]{images/fcnn-LDATime.png}
		\caption{Время обучения.}
	\end{subfigure}
	\caption{Результаты применения FCNN для LDA.}\label{fig:fcnn-lda-results}
\end{figure}

\begin{figure}[h!]
	\centering
	\begin{subfigure}{0.45\textwidth}
		\includegraphics[width=\textwidth]{images/fcnn-SVM.png}
		\caption{Качество классификации.}
	\end{subfigure}
	\begin{subfigure}{0.45\textwidth}
		\includegraphics[width=\textwidth]{images/fcnn-SVMTime.png}
		\caption{Время обучения.}
	\end{subfigure}
	\caption{Результаты применения FCNN для SVM.}\label{fig:fcnn-svm-results}
\end{figure}

\begin{figure}[h!]
	\centering
	\begin{subfigure}{0.45\textwidth}
		\includegraphics[width=\textwidth]{images/fcnn-randforest.png}
		\caption{Качество классификации.}
	\end{subfigure}
	\begin{subfigure}{0.45\textwidth}
		\includegraphics[width=\textwidth]{images/fcnn-randforestTime.png}
		\caption{Время обучения.}
	\end{subfigure}
	\caption{Результаты применения FCNN для Random forest.}\label{fig:fcnn-rf-results}
\end{figure}

\begin{figure}
	\centering
	\begin{subfigure}{0.45\textwidth}
		\includegraphics[width=\textwidth]{images/fcnn-gradboosting.png}
		\caption{Качество классификации.}
	\end{subfigure}
	\begin{subfigure}{0.45\textwidth}
		\includegraphics[width=\textwidth]{images/fcnn-gradboostingTime.png}
		\caption{Время обучения.}
	\end{subfigure}
	\caption{Результаты применения FCNN для GTB}\label{fig:fcnn-gtb-results}
\end{figure}

В целом, FCNN генерирует подвыборку немного большую, чем CNN, и тратит на это немного больше времени (в \cite{angiulli} указано, что FCNN имеет лучшую асимптотику, чем CNN, что может быть проверено на ещё больших наборах данных). Но это сполна компенсируется тем, что при сравнимых затратах времени на обучение базовых методов качество классификации оказывается у них намного выше, чем после CNN. FCNN всё равно приводит к ухудшению качества классификации, но по результатам экспериментов для \(k=1\) это ухудшение не превышает 1.9\%. Соответствующие данные приведены в таблице~\ref{table:fcnn-results} с использованием обозначений, аналогичных таблице~\ref{table:cnn-results}.

\begin{table}[h!]
    \centering
    \begin{tabular}{|c||c||c|c|}
    \cline{2-4}
    \multicolumn{1}{c||}{} & Сжатие \(T\) & \(k\)NN & LDA \\
    \hline \hline
	\input{fcnn-table1.txt}
\end{tabular}
\newline \vspace*{0.5cm} \newline
\begin{tabular}{|c||c|c|c|}
    \cline{2-4}
    \multicolumn{1}{c||}{} & SVM & Random forest & GTB \\
    \hline \hline
	\input{fcnn-table2.txt}
    \end{tabular}
    \caption{Сводная таблица результатов метода FCNN и базовых методов после применения FCNN}
    \label{table:fcnn-results}
\end{table}

\subsection{Class conditional instance selection}
CCIS \cite{marchiori} отличается от двух предыдущих методов тем, что он принадлежит к категории гибридных методов отбора эталонов. Он состоит из двух этапов: классово-условный выбор эталонов (CC) и прорежение (THIN).

Центральным для данного метода является граф классово-условных ближайших соседей \(G=(V,E)\) --- орграф, вершинам \(v\in V\) которого соответствуют элементы \(T\), а ребро \((a,b)\in E\), если \(b\) есть ближайший сосед \(a\) для некоторого класса. \(G\) разбивается на 2 подграфа: \(G_{wc}=(V,E_{wc})\), в котором есть лишь те рёбра из \(G\), которые инцидентны двум вершинам одного класса и \(G_{bc}=(V,E_{bc})\), в котором есть лишь те рёбра из \(G\), которые инцидентны двум вершинам разных классов. Полустепень захода в первом называется внутриклассовой полустепенью захода (within class in-degree), во втором --- междуклассовой полустепенью захода (between class in-degree).

Классово-условный отбор эталонов составляет подмножество \(S\) обучающей выборки \(T\), добавляя в него элементы по одному пока Leave-one-out-ошибка, полученная методом ближайшего соседа на \(S\), убывает и больше LOO-ошибки, полученной на \(T\). Порядок, в котором элементы \(T\) выбираются для добавления в \(S\), обусловлен теоретико-информационным критерием, зависящим от полустепеней захода элемента в \(G_{wc}\) и \(G_{bc}\), и характеризующим то, насколько глубоко объект находится внутри своего класса. В частности, из \(T\) удалятся все объекты, для которых междуклассовая полустепень захода больше внутриклассовой (такие элементы, скорее всего, лежат близко к границе классов.

Прорежение призвано отфильтровать \(S\), оставив объекты, наиболее важные для определения границы классов. Для этого из \(S\) выбирается подмножество \(S_f\), состоящее из объектов с положительной междуклассовой полустепенью захода в графе \(G^S\) для выборки \(S\) (эти объекты считаются важными для определения границы между классами). Затем \(S_f\) дополняется объектами из \(S\setminus S_f\) пока LOO-ошибка на \(T\) при использовании \(S_f\) в качестве обучающей выборки уменьшается.

\subsubsection*{Результаты использования CCIS}
Идея графа классово-условных ближайших соседей ориентирована на использование с методом одного ближайшего соседа, поэтому CCIS не имеет параметров. Таблица~\ref{table:ccis-results} содержит результаты работы CCIS и базовых методов после отбора им эталонов. В плюсы CCIS можно записать отличную редукцию обучающей выборки --- результат CCIS в нашей задаче имеет не более 10\% элементов исходной обучающей выборки, и, как следствие, большое ускорение тех базовых методов, которые изначально имели сколько-нибудь значительное время работы. \(k\)NN, случайный лес и градиентный бустинг ускорились более, чем в 10 раз, а SVM --- более, чем в 100, на всех каналах. Несомненным минусом CCIS является тот факт, что в нашей задаче он не смог обеспечить не только повышение качества классификации, но и удержание его на уровне данных из таблицы~\ref{table:base-all}. Данный результат приводит к мысли о том, что CCIS может быть полезен разве что для редукции сверхбольших выборок, с которыми невозможно работать в их полном составе. Хотя и в этом случае, скорее всего, есть более подходящие методы.

\begin{table}[h!]
    \centering
    \begin{tabular}{|c||c||c|c|}
    \cline{2-4}
    \multicolumn{1}{c||}{} & Сжатие \(T\) & \(k\)NN & LDA \\
    \hline \hline
	\input{ccis-table1.txt}
\end{tabular}
\newline \vspace*{0.5cm} \newline
\begin{tabular}{|c||c|c|c|}
    \cline{2-4}
    \multicolumn{1}{c||}{} & SVM & Random forest & GTB \\
    \hline \hline
	\input{ccis-table2.txt}
    \end{tabular}
    \caption{Сводная таблица результатов метода CCIS и базовых методов после применения CCIS}
    \label{table:ccis-results}
\end{table}


\section{Заключение}
В работе был дан обзор базовых методов машинного обучения и исследована их эффективность применительно к модельной задаче по распознаванию рекламы. На первом этапе произведен подбор основных параметров методов с целью получения оптимального качества классификации на исходных данных. Как и ожидалось, простые методы (\(k\)NN, LDA) показали не лучшее качество классификации, но зато они быстро работают. Более продвинутые методы работают медленнее и для них имеет смысл использовать селекцию признаков или эталонов (см. таблицу \ref{table:base-all}). 
\par
Далее были рассмотрены два метода отбора признаков: PCA и отбор по важности с помощью случайного леса (fandom forest selection или RFS). Оба метода оказались бесполезны в сочетании с \(k\)NN и LDA. Применение их вместе с SVM, gradient tree boosting (GTB) и случайным лесом дало схожие результаты: качество классификации SVM незначительно улучшилось (на \(1\%\) и менее), а качество классификации остальных методов до определённого момента слабо зависит от размерности редуцированного пространства признаков. При этом для всех методов, и в особенности для трудоёмких, просматривается линейная зависимость времени обучения от размерности пространства признаков. Таким образом, можно сказать, что редукция размерности имеет смысл для методов SVM, GTB и random forest, причём метод PCA позволяет максимально сократить размерность данных (как правило, до 50 переменных вместо исходных 228) с незначительными потерями качества классификации. Метод RFS в большинстве случаев позволяет оставить около 100 признаков. Его преимуществом является то, что часть признаков можно просто не извлекать при создании новых исходных данных, в то время как для использования PCA это всё равно придётся делать.

Среди методов отбора эталонов были рассмотрены, реализованы и использованы в экспериментах методы condensed nearest neighbor, fast condensed nearest neighbor и class conditional instance selection. Результат, достойный применения в практической деятельности, показал лишь метод FCNN, два других метода не смогли обеспечить приемлемое качество классификации базовых методов. В целом, большой проблемой метода отбора эталонов является то, что он представляет собой дополнительный шаг в процессе построения модели и требуемое для него время является критичным параметром. К сожалению, как отмечено в \cite{angiulli}, создатели методов отбора эталонов зачастую игнорируют этот критерий, из-за чего большая часть многообещающих методов с точки зрения сохранения потенциала обобщающей способности и редукции требует слишком большое время для работы. Возможно это одна из причин, по которой область instance selection не привлекает большой интерес в настоящее время.


\printbibliography
\end{document}

